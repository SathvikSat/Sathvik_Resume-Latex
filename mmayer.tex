%%%%%%%%%%%%%%%%%
% This is an example CV created using altacv.cls (v1.1.5, 1 December 2018) written by
% LianTze Lim (liantze@gmail.com), based on the
% Cv created by BusinessInsider at http://www.businessinsider.my/a-sample-resume-for-marissa-mayer-2016-7/?r=US&IR=T
%
%% It may be distributed and/or modified under the
%% conditions of the LaTeX Project Public License, either version 1.3
%% of this license or (at your option) any later version.
%% The latest version of this license is in
%%    http://www.latex-project.org/lppl.txt
%% and version 1.3 or later is part of all distributions of LaTeX
%% version 2003/12/01 or later.
%%%%%%%%%%%%%%%%

%% If you are using \orcid or academicons
%% icons, make sure you have the academicons
%% option here, and compile with XeLaTeX
%% or LuaLaTeX.
% \documentclass[10pt,a4paper,academicons]{altacv}

%% Use the "normalphoto" option if you want a normal photo instead of cropped to a circle
% \documentclass[10pt,a4paper,normalphoto]{altacv}

\documentclass[10pt,a4paper,ragged2e]{altacv}

%% AltaCV uses the fontawesome and academicon fonts
%% and packages.
%% See texdoc.net/pkg/fontawecome and http://texdoc.net/pkg/academicons for full list of symbols. You MUST compile with XeLaTeX or LuaLaTeX if you want to use academicons.

% Change the page layout if you need to
\geometry{left=2cm,right=10cm,marginparwidth=6.8cm,marginparsep=1.2cm,top=1.25cm,bottom=1.25cm}

% Change the font if you want to, depending on whether
% you're using pdflatex or xelatex/lualatex
\ifxetexorluatex
  % If using xelatex or lualatex:
  \setmainfont{Carlito}
\else
  % If using pdflatex:
  \usepackage[utf8]{inputenc}
  \usepackage[T1]{fontenc}
  \usepackage[default]{lato}
\fi

% Change the colours if you want to
\definecolor{VividPurple}{HTML}{000000}
\definecolor{SlateGrey}{HTML}{2E2E2E}
\definecolor{LightGrey}{HTML}{2E2E2E}
\colorlet{heading}{VividPurple}
\colorlet{accent}{VividPurple}
\colorlet{emphasis}{SlateGrey}
\colorlet{body}{LightGrey}

% Change the bullets for itemize and rating marker
% for \cvskill if you want to
\renewcommand{\itemmarker}{{\small\textbullet}}
\renewcommand{\ratingmarker}{\faCircle}

%% sample.bib contains your publications
\addbibresource{sample.bib}

\begin{document}
\name{Sathvik S}
\tagline{Electronics and Communication Undergraduate}
% Cropped to square from https://en.wikipedia.org/wiki/Marissa_Mayer#/media/File:Marissa_Mayer_May_2014_(cropped).jpg, CC-BY 2.0
%\photo{3.3cm}{profile.jpg}
\personalinfo{%
  % Not all of these are required!
  % You can add your own with \printinfo{symbol}{detail}
  \email{sathvik@ieee.org}
  \phone{(+91) 7892048017}
%  \mailaddress{Address, Street, 00000 County}
  \location{Mysore, IN}
%  \homepage{marissamayr.tumblr.com/}
%  \twitter{@marissamayer}
  \linkedin{linkedin.com/in/sathvik-s-setty/}
   \github{github.com/SathvikSat} % I'm just making this up though.
%   \orcid{orcid.org/0000-0000-0000-0000} % Obviously making this up too. If you want to use this field (and also other academicons symbols), add "academicons" option to \documentclass{altacv}
}

%% Make the header extend all the way to the right, if you want.
\begin{fullwidth}
\makecvheader
\end{fullwidth}

%% Depending on your tastes, you may want to make fonts of itemize environments slightly smaller
\AtBeginEnvironment{itemize}{\small}

%% Provide the file name containing the sidebar contents as an optional parameter to \cvsection.
%% You can always just use \marginpar{...} if you do
%% not need to align the top of the contents to any
%% \cvsection title in the "main" bar.
\cvsection[page1sidebar]{Experience / Internships}
\cvevent{Summer Intern}{National Institute of Technology Karnataka - Dept. of ECE}{Jun. 2019 - Jul. 2019}{Surathkal, IN}
\begin{itemize}
\item Notably Involved in building a wearable device for KMPC Hospital".
\item Developed a single error correcting Hamming Code, learnt concepts of LDPC Codes and BCH/ReedSolomon Codes. Also, got a good exposure towards working in a research field".
\item Engaged in working on Intel Galileo gen 2 boards at a precursory level, to get an exposure.
%\smallskip
\end{itemize}

\divider

\cvevent{Embedded Intern}{TIF- LABS Pvt Ltd. (Robocraze)}{Dec. 2017 - Jan. 2018}{Bangalore, IN}
\begin{itemize}
\item As an intern got familiarised into technologies like Embedded Systems, 
and Internet of Things using MQTT and HTTP.
\item Completed basic hands-on training on Arduino, NodeMCU and ESP32. 
\item Was majorly involved in a project on developing experiments for 'IOTIF Manual'
\item Built and tested IOTIF trainer Kit and got introduced to Raspberry Pi.
%\smallskip
\end{itemize}

%\divider


\cvsection{Projects}
\cvevent {NIE Summer of Codes 5.0}{}{}{}
\begin{itemize}
\item Projects based on Web server, Bluetooth SD card reader, Remote Temperature logger, OTA, Web Client, MQTT \& HTTP Protocol, etc. built using ESP 32 and SD card module.
%\smallskip
\end{itemize}

\divider

\cvevent  {Wireless Data Transmission using LASER}{}{}{}
\begin{itemize}
\item Stream of data bits were transmitted via LASER at the transmitter end
\item An Light Dependent Resistor was used as a photo detector at the reception
\item Arduino Nano was used at transmission and reception end
\item The goal was to demonstrate a  prototype for Optical Communication
%\smallskip
\end{itemize}

\divider

\cvevent  {Gradient Descent from Scratch}{}{}{}
\begin{itemize}
\item Implemented Gradient Descent Algorithm to minimize the Mean Square Error 
\item Linear Regression - House Price Prediction
\item Logistic Regression - Breast Cancer Classification 
%\smallskip
\end{itemize}

\divider

\cvevent  {Vehicle to Infrastructure(V2X) Communication}{}{}{}
\begin{itemize}
\item Vehicle to Infrastructure communication  using MQTT Protocol
\item Raspberry Pi was used at Vehicle \& Infrastructure end. Cloud MQTT was used as broker.
%\smallskip
\end{itemize}

\divider 
\newpage

\cvsection{leadership / Volunteership }
%\smallskip
\cvevent{Editor-in-Chief}{NIE IEEE STUDENT BRANCH (NISB)}{Jun. 2018 - Jun. 2019}{Mysore, IN}
\begin{itemize}
\item Organised BIBLUS, A National Level Technical Paper Presentation Contest sponsored by TEQIP-III
\item Released the fifth edition of Newsletter, MANAS (I and II).
\item Released the fifth edition of Annual Magazine, JIJNASA.
\end{itemize}

\divider

\cvevent  {Resource Manager}{NISB}{Oct. 2017 - Jun. 2018}{Mysore, IN}
\begin{itemize}
\item Maintained a record of all the pre-post events/Workshops 
\item Motivated  students to attend for the same.
\end{itemize}
\divider

\cvevent {ECE Representative}{Dept. of ECE, NIE }{Sep. 2016 - Aug. 2017} {}{}
\begin{itemize}
\item Organised and coordinated Department fest (NIECFEST)
\item Conducted technical events on Arduino and C++
\end{itemize}
\divider
\begin{itemize}
\item As an active memeber of NISB - Won a Largest Outstanding Student Branch Award with team
\item Organised \& co-ordinated biggest technical fests of NIE - 'Adroit' \& 'Ankura'
\item Organized over 70+ events, 20+ workshops with the team
\item Conducted sessions on Introduction to Python as a part of Women In Engineering weekly meet-ups
\item Conducted Basic Electronics Sessions \& brain storming Electronika 9.0 as an Execom
\item Conducted Arduino Focus Group Sessions.
\end{itemize}

\cvsection{MANAGEMENT SKILLS}
\begin{itemize}
\item Content Writing \& Editing
\item Team Leadership, Organization Leadership
\item Event Management \& Programming
\item Public Speaking
\item Sponsorship 
%\smallskip
\end{itemize}

\cvsection{Course Electives}
\begin{itemize}
\item Artificial Intelligence and Neural Networks
\item System Verilog
\end{itemize}
%\divider

\cvsection{Membership}
\cvevent{IEEE - Institute of Electrical and Electronics Engineers} {Aug. 2016 - Present}{}{}
%\divider






% \cvevent{Product Engineer}{Google}{23 June 1999 -- 2001}{Palo Alto, CA}

% \begin{itemize}
% \item Joined the company as employe \#20 and female employee \#1
% \item Developed targeted advertisement in order to use user's search queries and show them related ads
% \end{itemize}

%\cvsection{A Day of My Life}

% Adapted from @Jake's answer from http://tex.stackexchange.com/a/82729/226
% \wheelchart{outer radius}{inner radius}{
% comma-separated list of value/text width/color/detail}
% Some ad-hoc tweaking to adjust the labels so that they don't overlap
% \wheelchart{1.5cm}{0.5cm}{%
%   10/10em/accent!30/Sleeping \& dreaming about work,
%   25/9em/accent!60/Public resolving issues with Yahoo!\ investors,
%   5/13em/accent!10/\footnotesize\\[1ex]New York \& San Francisco Ballet Jawbone board member,
%   20/15em/accent!40/Spending time with family,
%   5/8em/accent!20/\footnotesize Business development for Yahoo!\ after the Verizon acquisition,
%   30/9em/accent/Showing Yahoo!\ employees that their work has meaning,
%   5/8em/accent!20/Baking cupcakes
% }

\clearpage

% \cvsection[page2sidebar]{Publications}

\nocite{*}

% \printbibliography[heading=pubtype,title={\printinfo{\faBook}{Books}},type=book]

% \divider

% \printbibliography[heading=pubtype,title={\printinfo{\faFileTextO}{Journal Articles}}, type=article]

% \divider

% \printbibliography[heading=pubtype,title={\printinfo{\faGroup}{Conference Proceedings}},type=inproceedings]

% %% If the NEXT page doesn't start with a \cvsection but you'd
% %% still like to add a sidebar, then use this command on THIS
% %% page to add it. The optional argument lets you pull up the
% %% sidebar a bit so that it looks aligned with the top of the
% %% main column.
% % \addnextpagesidebar[-1ex]{page3sidebar}


\end{document}
